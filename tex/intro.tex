\newpage
\section{Введение}
    Матричное исчисление имеет широкое применение в различных отраслях науки. Проверка того, 
    обратима ли данная матрица над полем или кольцом, и вычисление обратной матрицы
    являются классическими математическими задачами.


    В работе рассматриваются матрицы, элементы которых принадлежат 
    кольцу скалярных линейных дифференциальных операторов над
    дифференциальным полем, т.е. над полем, снабжённым операцией дифференцирования.
    В этом случае вместо понятия обратимости матрицы используется понятие 
    \emph{унимодулярности} матрицы, которое и будет использовано далее.


    Такие матрицы возникают при работе с линейными дифференциальными системами, 
    которые в свою очередь возникают во многих приложениях, таких как 
    системы многих тел, модели электрических цепей, моделирование роботов, 
    механические системы и т.д.


    Для решения задачи проверки унимодулярности операторной матрицы и построения 
    обратной матрицы могут использоваться некоторые известные алгоритмы преобразования 
    матриц с помощью обратимых операций по строкам или столбцам. Такими являются, 
    например, алгоритмы построения форм Эрмита или Джекобсона, а также алгоритм 
    Row-Reduction.  


    Для работы с операторными матрицами, как правило, используются специализированные 
    системы компьютерной алгебры. Такой системой называется программное обеспечение, 
    предназанченное для выполнения математических операций и алгебраических 
    вычислений с символьными выражениями вместо числовых значений. 

    
    На данный момент существует большое количество систем компьютерной алгебры. Некоторые из них 
    являются платными, например, Maple и Mathematica, а некоторые, такие как 
    Maxima, SageMath или библиотека SymPy для языка Python, распространяются 
    свободно. В работе используется система SageMath по причине своей доступности и 
    идентичного языку Python синтаксису.
\newpage
\section{Постановка задач}
    В курсовой работе требовалось:
    \begin{enumerate}
        \item Изучить систему компьютерной алгебры SageMath.
        \item Изучить существующие алгоритмы проверки унимодулярности.
        \item Реализовать алгоритм проверки унимодулярности матрицы 
        дифференциальных операторов в системе компютерной алгебры SageMаth.
    \end{enumerate}